\chapter{Shapelet learning}

\section{Preliminaries}

\section{Prediction}

As for the approaches from the previous two chapters, we evaluate how well the shapelet learning model predicts the health state of proportional valves by measuring the area under the \acrshort{roc} curve which figure~\ref{fig:roc-shapelet-learning} depicts.

\begin{figure}
  \includegraphics[width=\textwidth]{roc_shapelet_learning.png}
  \caption{The \acrshort{roc} curve for the shapelet learning model.}
  \label{fig:roc-shapelet-learning}
\end{figure}

\section{Conclusion}

Shapelet learning leads to a model which predicts proportional valve health state well, although not as well as logistic regression on manually extracted variables.
However, it does so by only considering the pressure sensor waveform, whereas the manually extracted variables come from the current, pressure, and temperature sensor waveforms.

Both the implementation of shapelet learning in pyts and the implementation in tslearn are lacking.
Although both implementations allow the user to split data in batches, they do not allow the user to stream the data from a database or file system, thereby defeating the purpose of batching.
Furthermore, the implementation in pyts does not allow parallel processing.
Finally, the implementation in tslearn consumes exorbitant amounts of memory for shapelets of non-trivial length.
We conclude that both implementations need to further mature to see use in production applications.
