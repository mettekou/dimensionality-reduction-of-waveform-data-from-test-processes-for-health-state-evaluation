\chapter{Introduction}

Because we statistically analyze data in this thesis, it is opportune to first introduce the data at hand.
It originates from an automotive supplier company which develops, produces, and tests automotive power transmission systems such as \acrlongpl{dct} (\acrshortpl{dct}) for performance vehicles.
In addition, it does so for most components of its \acrshortpl{dct}, such as clutches and hydraulic subsystems.
The test processes for some of the simplest components of these hydraulic subsystems, namely proportional valves, produce the data which we analyze in this thesis.
As the company produces these proportional valves in high quantities, the decision to analyze their data makes sense from both the statistical and the engineering perspective.

% Add figure of proportional valve
Let us now turn our attention to the subject of the test process data.
Proportional valves are electromechanical fluid pressure regulation devices common in hydraulic systems \citep{DBLP:journals/tii/MazaevCH21}.
The proportional valves we consider here contain a solenoid, which is an electromagnet formed by a coil.
To regulate pressure in the proportional valve, we send an electrical current through this coil and the resulting magnetic field attracts a plunger.
The plunger movement can be linear to proportionally regulate fluid pressure or sudden to either allow maximum fluid pressure or none whatsoever.
In this dissertation we only concern ourselves with proportional valves with a maximum pressure of 35 bar.
This decision is not due to our methods not being generally applicable, but rather due to the data set we have available.

% See Automotive Power Transmission Systems
%Simple mechatronic devices such as solenoid valves do not stand alone, but serve as the components of more complex mechatronic devices.

The hydraulic systems which incorporate the proportional valves are even more diverse than the valves themselves.
This should come as no surprise, as pressure regulation is arguably the main method of control in such systems.

It is also necessary to describe the test processes for proportional valves here.
During test processes for mechatronic devices, soft real-time systems and field-programmable gate arrays (FPGAs) sample tens of sensors mounted on a \acrlong{dut} (\acrshort{dut}) and test bench at rates exceeding 1 kHz.
Given that these test processes last several minutes for all but the simplest of mechatronic devices, the resulting waveforms consist of hundreds of thousands of samples each.
Because they precisely capture the \acrshort{dut}’s response to the test process, along with any edge conditions stemming from the test bench, the waveforms are stored for analysis at the end of the test process, together with categorical data, such as which components were used during the assembly of the DUT.\@

The goal of this analysis depends on the environment in which it takes place.
In a development environment, the DUTs are iterative improvements over an initial product design.
Product engineers either inspect the waveforms from these test processes visually or they write software to analyze them via numerical mathematics. They find features of the waveforms which are the result of the DUT’s behaviour deviating fromthe norm. Software engineers then adapt the product engineers’ work for use in a production environment.
The goal here is no longer to develop an understanding of the product’s behaviour, but to detect deviations in the DUT’s behaviour immediately after the test process.


\section{Problem statement}

A problem with the current approach is the workload for software engineers.

Furthermore, More errors due to manual numerical mathematics/algorithms

No standard procedure

It is clear that the analysis is rigid: the features of interest to product engineers during development are the ones which are evaluated in production.
Due to the small amount of devices for which the waveforms are analyzed during development, it is also biased.
Some defects are rare enough to evade detection during development, but common enough to lead to considerable losses in production.
When these defects are detected in production, product and process engineers have to revise the analysis and software engineershave to adapt their software.
Production then has to be stopped to implement the adapted software.

\section{Research objectives}

For a more flexible and unbiased analysis, we propose the use of a statistical model, both in the development
and production environment. It can take both the high-dimensional waveform test data and the categorical
assembly data as inputs. The student can opt for a traditional statistical model or a machine learning model.
Either way, he can subject it to several research questions of interest to us.

\begin{itemize}
      \item Compare various automatic data reduction methods for waveforms as
            preprocessing for a regression model
      \item Given the waveforms characterizing a development test process for a few DUTs, can a statistical
            model find features of interest? Some input from product engineers, such as the features they deem
            interesting, can be used as a stepping stone if necessary.
      \item Compare regression model using data reduction on waveforms to
            regression model using features extracted by domain experts
      \item Given the waveforms characterizing a production test process for many DUTs, can a statistical model
            find features of interest not yet found during development? Similarly, can it eliminate features found
            in development no longer relevant during production?
      \item Developing a standard procedure
\end{itemize}

\section{Outline}

The next chapter is on related work.
